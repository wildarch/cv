%%%%%%%%%%%%%%%%%%%%%%%%%%%%%%%%%%%%%%%%%
% "ModernCV" CV and Cover Letter
% LaTeX Template
% Version 1.1 (9/12/12)
%
% This template has been downloaded from:
% http://www.LaTeXTemplates.com
%
% Original author:
% Xavier Danaux (xdanaux@gmail.com)
%
% License:
% CC BY-NC-SA 3.0 (http://creativecommons.org/licenses/by-nc-sa/3.0/)
%
% Important note:
% This template requires the moderncv.cls and .sty files to be in the same 
% directory as this .tex file. These files provide the resume style and themes 
% used for structuring the document.
%
%%%%%%%%%%%%%%%%%%%%%%%%%%%%%%%%%%%%%%%%%

%----------------------------------------------------------------------------------------
%	PACKAGES AND OTHER DOCUMENT CONFIGURATIONS
%----------------------------------------------------------------------------------------

\documentclass[10pt,a4paper,sans,colorlinks,linkcolor=color1,urlcolor=color1]{moderncv} % Font sizes: 10, 11, or 12; paper sizes: a4paper, letterpaper, a5paper, legalpaper, executivepaper or landscape; font families: sans or roman

\moderncvstyle{classic} % CV theme - options include: 'casual' (default), 'classic', 'oldstyle' and 'banking'
\moderncvcolor{blue} % CV color - options include: 'blue' (default), 'orange', 'green', 'red', 'purple', 'grey' and 'black'

\usepackage[scale=0.93]{geometry} % Reduce document margins
%\setlength{\hintscolumnwidth}{3cm} % Uncomment to change the width of the dates column
%\setlength{\makecvtitlenamewidth}{10cm} % For the 'classic' style, uncomment to adjust the width of the space allocated to your name


%----------------------------------------------------------------------------------------
%	NAME AND CONTACT INFORMATION SECTION
%----------------------------------------------------------------------------------------
\firstname{Daan} % Your first name
\familyname{de Graaf} % Your last name

% All information in this block is optional, comment out any lines you don't need
%\title{Curriculum Vitae}
\address{Tongelresestraat 92A}{Eindhoven, Netherlands 5613 DN}
\mobile{(+31) 6 232 571 62}
%\phone{(000) 111 1112}
%\fax{(000) 111 1113}
\email{daandegraaf9@gmail.com}
%\homepage{staff.org.edu/~jsmith}{staff.org.edu/$\sim$jsmith} % The first argument is %the url for the clickable link, the second argument is the url displayed in the %template - this allows special characters to be displayed such as the tilde in this %example
%\homepage{https://wildarch.dev}{wildarch.dev}
\extrainfo{
    Homepage: \href{https://wildarch.dev}{wildarch.dev}\\
    Github: \href{https://github.com/wildarch}{wildarch}\\
    LinkedIn: \href{https://www.linkedin.com/in/daan-de-graaf}{daan-de-graaf}
}
\photo[70pt][0.4pt]{profile} % The first bracket is the picture height, the second is %the thickness of the frame around the picture (0pt for no frame)


%----------------------------------------------------------------------------------------
\begin{document}

\makecvtitle % Print the CV title

%----------------------------------------------------------------------------------------
%	EDUCATION SECTION
%----------------------------------------------------------------------------------------

\section{Education}
\cventry{09/2021 -- 08/2023}{Computer Engineering MSc}{Delft University of Technology}{Delft}{\textit{expected Cum laude}}{
    Core courses on Computer Architecture, Processor Design and Heterogeneous Computing.
    Specialization courses in Compiler Construction, Distributed \& Concurrent Program Analysis, and Data-Intensive Systems Engineering.
    Some courses in Databases and Functional Programming taken at TU Eindhoven.
    GPA: 9.0/10
}
\cventry{09/2015 -- 07/2018}{Computer Science BSc}{Eindhoven University of Technology}{Eindhoven}{Cum laude}{
    Computer Science bachelor with electives in Security, Artificial Intelligence and Computer Architecture. GPA: 8.55/10.0
}

\cventry{09/2009 -- 06/2015}{Pre University Education}{Gymnasium Camphusianum}{Gorinchem}{Cum laude}{
    Secondary education in preparation for academic studies, including ancient Greek and advanced calculus. GPA: 8.36/10.0
}

%---------------------------------------------------------------------------------------
%	WORK EXPERIENCE SECTION
%----------------------------------------------------------------------------------------
\section{Work}
\cventry{06/2023 -- present}{PhD Candidate on Programmability of Graph Databases}{Eindhoven University of Technology}{Eindhoven}{\href{https://research.tue.nl/en/persons/daan-de-graaf}{https://research.tue.nl/en/persons/daan-de-graaf}}{
    My research currently focuses on support for user-defined graph algorithms in graph databases.
    I am developing Graphalg, a domain-specific language for writing graph algorithms in the language of linear algebra that is both easy to use and highly optimizable.
    Fully integrated into the AvantGraph database, it enables graph databases to become platforms for efficient large-scale data analysis.
}
\cventry{09/2019 -- 07/2021}{Software Engineer}{Google LLC}{London}{}{
    As part of the Display Ads Monetization team, developed the pipeline for calculation and disbursement of advertising revenue to AdSense and YouTube partners.
    The system I personally designed and built processed well over 1 billion USD monthly, and was a mission-critical part of the pipeline.
    In response to concerns about exceeding our time budget, I spearheaded an initiative to speed up our pipeline, and reduced execution time six-fold.
    Next to my core responsibilities, I have also taught introductory classes to the \href{https://www.rust-lang.org/}{Rust programming language} and contributed to the \href{https://v8.dev/}{V8 JavaScript compiler} that powers Google Chrome.
}
\cventry{09/2018 -- 05/2019}{Designer}{Prodrive Technologies B.V.}{Eindhoven}{\href{https://prodrive-technologies.com}{prodrive-technologies.com}}{
    After 2 months as an Embedded Systems Engineer, I switched to the newly created Developer Infrastructure team to assess the viability of the Rust programming language for use at Prodrive.
    I helped various teams in the Embedded and Networking domains to improve the reliability of their products and increase developer productivity.
}

\section{Notable Projects}
\cventry{02/2023 -- 08/2023}{MSc Thesis: Efficient Processing of User-Provided Graph Algorithms in a Graph Database}{TU Delft / \mbox{TU Eindhoven}}{Delft}{\href{http://resolver.tudelft.nl/uuid:1b5f0236-87f2-41f0-8af8-0911cb3b4d54}{http://resolver.tudelft.nl/uuid:1b5f0236-87f2-41f0-8af8-0911cb3b4d54}}{
    In this joint project between TU Delft and TU Eindhoven, I researched how to design a user-friendly programming language for expressing graph algorithms that can be compiled and executed efficiently inside the \href{https://avantgraph.io/}{AvantGraph Database}.
    I developed the compiler for this language, which optimizes and transforms imperative programs into a query operator graph.
    My thesis project allows the embedding of graph algorithms directly into graph queries, which can then be holistically optimized and executed.
}
\cventry{11/2022 -- 02/2023}{Superg}{TU Eindhoven}{Eindhoven}{\href{https://github.com/wildarch/superg}{github.com/wildarch/superg}}{
    Built modern equivalents of the \href{http://miranda.org.uk/}{Miranda} and \href{https://users.ece.cmu.edu/~koopman/tigre/index.html}{TIGRE} execution engines for functional programming languages.
    Thanks to the use of a recently-proposed compilation algorithm, the worst-case performance is exponentially better than previous Combinator Reduction Engines.
    We show that while TIGRE outperformed Miranda significantly at the time of publication, on modern x86 CPUs it suffers from frequent cache invalidation, and is not a good choice anymore.
    Final course grade 10/10.
}

\cventry{04/2022 -- 02/2023}{ByzzFuzz}{TU Delft}{Delft}{\href{https://doi.org/10.1145/3586053}{doi.org/10.1145/3586053}}{
    Designed a fuzz testing harness for the Tendermint blockchain.
    I demonstrated a mismatch between the consensus protocol and message delivery subsystem, which can cause clusters to deadlock if certain messages are dropped.
    Our work received the ACM SIGPLAN Distinguished Paper award at \href{https://2023.splashcon.org/details/splash-2023-oopsla/27/Randomized-Testing-of-Byzantine-Fault-Tolerant-Algorithms}{OOPSLA 2023}.
}

\cventry{02/2022 -- 04/2022}{Compiled Selection Predicates in the AvantGraph Database}{TU Eindhoven}{Eindhoven}{}{
    Designed a Just-in-time compilation scheme for the AvantGraph database that combines performance with maintainability.
    The JIT compiler, based on LLVM, accepts ordinary C++ functions and applies runtime specialization to reduce branching and pointer indirection.
    I implemented selection predicate compilation using the JIT and demonstrated performance improvements in multiple queries from the \href{https://github.com/ldbc/lsqb}{LSQB} benchmark.
    Final course grade 9/10.
}

\newpage

\section{Honors}
\cventry{06/2016 -- 09/2018}{Honors Program High Tech Systems}{TU Eindhoven}{Eindhoven}{}{
    Special program for the top 100 students of the university to take on an extra project next to their studies.
    Built a low-latency drone position-tracking system based on image recognition.
    At our industry partner Avular, I ported their position tracking system from Matlab to C++ and CUDA to run it locally on the drone, achieving a 4x speedup.
    Received the rare "Excellent" distinction for my efforts (highest awarded).
}

\section{Competences}
\cvitem{Languages}{C/C++, Rust, Java, SQL, Python, \LaTeX, VHDL}
\cvitem{Tech. skills}{
    Compiler design,
    Engineering data-intensive systems,
    Performance engineering,
    Large-scale software development
}
\cvitem{Soft skills}{
    Academic writing,
    Public speaking,
    Project management
}

\cvitem{Technologies}{
    Linux,
    LLVM,
    ANTLR,
    Vim,
    Bazel,
    NVIDIA CUDA,
    Docker
}

%----------------------------------------------------------------------------------------
%	LANGUAGES SECTION
%----------------------------------------------------------------------------------------
\section{Languages}
\begin{small}
    \hspace{2.4cm}
    \textbf{Dutch:} Native speaker
    \hspace{0.4cm}
    \textbf{English:} Advanced
\end{small}

\section{Board experience}
\cventry{06/2017 -- 09/2018}{Chairman}{Scala}{Eindhoven}{\href{https://studentencultuur.nl}{studentencultuur.nl}}{
    Represent a collective of 9 cultural student associations with over 1000 members.
    Tasks include organizing large events, policy-making and negotiating board grant agreements.
}

\cventry{09/2016 11/2017}{President}{AEGEE-Eindhoven}{Eindhoven}{\href{https://aegee.org}{aegee.org}}{
    Represent the local AEGEE antenna both locally and internationally.
    Main event organizer, and coordinator for the move to a new building.
}


%----------------------------------------------------------------------------------------
%	INTERESTS SECTION
%----------------------------------------------------------------------------------------
\section{Interests}
\cvitem{Non-Western culture}{
    I am interested in how different cultural backgrounds shape people's lives and opinions.
    In recent years I have traveled to Turkey, Moldova, Azerbaijan, Vietnam and Hong Kong, and I hope to visit mainland China soon.
}

\cvitem{Cooking}{
    I enjoy trying out new recipes and learning the science behind cooking to make better food.
    In particular, I am interested in Asian cuisine and bread-making.
    My collection of recipes is available on \href{https://github.com/wildarch/mono/tree/main/recipes/collections}{GitHub}.
}

\cvitem{Climbing}{
    I like indoor bouldering and top-rope climbing. I hope to try outdoor climbing in the future.
}

\end{document}
